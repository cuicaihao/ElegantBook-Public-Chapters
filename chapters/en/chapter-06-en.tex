\chapter{Mathematical Tools}

This appendix covers some of the basic mathematics used in econometrics. We briefly discuss the properties of summation operators, study the properties of linear and some nonlinear equations, and review the ratios and percentages. We also introduce some special functions that are common in econometrics applications, including quadratic functions and natural logarithms. The first four sections require only basic algebraic techniques. The fifth section briefly reviews differential Calculus Although Calculus is not necessary to understand much of this book, it is used in some of the end-of-chapter appendices and in some of the more advanced topics in part 3.

\section{Summation Operator and Description Statistics}

\textbf{Summation Operator} is an abbreviation used to express the summation of numbers, it plays an important role in statistics and econometrics analysis. If $\{x_i: i=1, 2, \ldots, n\}$ is a sequence of $n$ numbers, the summation of the $n$ numbers is:

\begin{equation}
\sum_{i=1}^n x_i \equiv x_1 + x_2 +\cdots + x_n
\end{equation}
